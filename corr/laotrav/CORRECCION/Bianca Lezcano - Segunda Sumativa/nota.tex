\documentclass{article}
\usepackage{mathptmx}
\usepackage{amsfonts, amsmath, amssymb}

\begin{document}
\section*{P2}{
  \itemize{
  \item tipos de funciones que no coinciden.
  \item Se olvidó de pasar la variable a imprimir a printf (escalera)
  \item Todos sus arrays son de 100.
  \item A la funcion escalera pasa un entero en lugar de un vector.
  \item Usa cada vez a la funcion como una variable.
  \item SU FORMA DE HALLAR LA CANTIDAD DE VECES QUE SE REPITEN LOS DADOS NO TIENE SENTIDO.
  \item La función Mayor no tiene sentido.
  \item Tiene una variable global que se llama Contador, y otra en una función llamada Contador
  \item En ningún momento se llama a la función rand()
  \item Nunca llama a escalera.
  \item Su código no tiene sentido, salvo la función escalera.
  \item Nunca llama a escalera.
  \item Crea un vector Vector con dos elementos en lugar de tres, 
    y accede al tercer elemeto, lo cual puede romper mi compu.
  \item La función Sumatoria no tiene sentido.
  \item 
  }
}
\end{document}

