\documentclass{article}
\usepackage[a4paper, total={6in, 8in}]{geometry}
\usepackage{graphicx}
\usepackage{mathptmx}

\begin{document}
\begin{titlepage}
  \begin{center}
      \vspace*{1cm}

      \Huge
      \textbf{Primeros Gobiernos Independientes}

      \vspace{0.5cm}
      \LARGE
      \vspace{1.5cm}

      \textbf{Jonathan Bray, Ever Ortega, Cp, Amira Balbuena}

      \vfill

      Historia y Geografía

      \vspace{0.8cm}


      \Large
      Colegio Técnico Nacional\\
      Especialidad: Informática\\
      Profesora: Mirian Montanía
      14/08/2024

  \end{center}
\end{titlepage}

  \Large
  %\noindent Primeros gobiernos independientes\dotfill2\\
  \noindent Nota del 20 de julio de 1810\dotfill2\\
  \noindent Triunvirato o Gobierno Provisorio\dotfill2\\
  \noindent Congreso de 1811. Primer Congreso Nacional\dotfill2\\
  \noindent Junta Superior Gubernativa\dotfill3\\
  \noindent Obras de la Junta Superior Gubernativa\dotfill3\\
  \noindent Educación\dotfill4\\
  \noindent Economía\dotfill4\\
  \noindent Misión de Nicolás Herrera\dotfill5\\
  \noindent Segundo Congreso Nacional, 1813\dotfill5\\
  \noindent El Reglamento de Gobierno, 1813\dotfill5\\
  \noindent El Primer Consulado\dotfill6

  \normalsize
  \pagebreak
  
  \section*{\large Primeros gobiernos independientes}
  \hrule
  \vspace{0.5cm}
  \subsection*{Nota del 20 de julio de 1810}

A partir de la separación del gobernador
Bernardo de Velasco del gobierno,
decisión tomada por el Congreso
Nacional de 1811, los primeros gobiernos
que se sucedieron en el periodo
independiente fueron ocupados por
paraguayos comprometidos con el
pueblo en salvaguardar la soberanía
del Paraguay y luchar por la
independencia que habían alcanzado. 
Nota del 20 de julio de 1810 julio
de enviado es conocido como Nota
del 20 de lullo de 1810 dejaba claro
que la Provincia del Paraguay estaría
gobernada por sí misma y que la Junta
de Buenos Aires no iba a ejercer ningún 
tipo de influencia, autori-dad o poder
sobre ella. Poco después, el doctor
Francia se retiró de la Junta Superior
Gubernativa
debido a su disgusto por el actuar de
algunos vocales de dicha Junta.

  \subsection*{Triunvirato o Gobierno Provisorio}
En medio de grandes dificultades,
la Junta realió un en la realización de sus planes. En 1813, la Junta convocó a otro Congreso Nacional. 

Triunvirato o Gobierno Provisorio.

El resultado de la acción revolucionaria fue la formación
del primer gobierno de la época independiente del Paraguay. Para 
el efecto fueron invitados a formar parte del Triunvirato los 
representantes de las tendencias políticas en pugna: el capitán
Juan Valeriano (de) Zeballos, en representación de los
terratenientes; el doctor José Gaspar Rodríguez de Francia,
de las ideas populares; y don Benardo de Velasco, de la corriente
españolista.

Por medio de un bando de fecha 9 de junio de 1811, se anunció
a los habitantes la causa de la separación de Bernardo de Velasco
del Triunvirato, donde se hacía entrence el entendimiento de este
con tropas de colonos pon gueses que le facilitarían retornar al
poder. El Triunvirato duró un mes y un día, del 16 de mayo al 17 
de junio de 1811, y se convocó a un Congreso para decidir la nuera
forma de gobierno del Paraguay.


\section*{Congreso de 1811. Primer Congreso Nacional}
Para decidir la forma de gobierno que iba a regir en el
Paraguay se convocó a un Congreso del 17 al 20 de junio de 1811;
se reunieron en la Casa de los Gobernadores, los representantes
de todo el país para nombrar un gobierno definitivo. Y para evitar
cualquier tipo de levantamiento en contra del Congreso y debilitar 
las pretensiones de los portenistas, el doctor Francia y Zeballos 
ordenaron el arresto de vatios líderes que estaban a favor del
sometimiento a la autoridad de Buenos Aires; esto lo hicieron dos
sentaras antes del inicio de las sesiones del Congreso.

Para este Congreso el doctor Francia procuró que sostuvieran la
mayor cantidad posible de representantes populares de las villas,
pueblos y partidos del interior del país; sin embargo, esto no fue
posible, puesto que la mayoría de los 251 delegados que asistieron
pertenecían a la clase alta, a la élite criolla de la ciudad de Asunción. A pesar de ello, Francia imbuido
en las ideas de la Ilustración, propició y mantuvo en las Asambleas 
un ambiente de igualdad en su discurso de apertura de las sesiones. 
El Congreso ---integrado por el doctor Francia, Pedro Juan Cavallero,
Francisco Javier Bogarín y Fernando de la Mora como vocales--- nombró
presidente de la Junta Superior Gubernativa a Fulgencio Yegros, 
quien junto con Cavallero representó al estamento militar, Bogarín 
al estamento del clero, Fernando de la Mora a la burguesía comercial
asuncena, y el doctor Joseé Gaspar de Francia en cierta manera fue
el representante de los pueblos del interior de país. 

En su discurso, el doctor Francia dio a conocer los sucesos de la 
independencia y de la deposición del gobernador Velasco; además dejó
sentada la postura clara de su ideología, sobre la libertad de los 
individuos, afirmando que todos los hombres nacen libres y que 
mientras vivan sujetos bajo una autoridad vivirán oprimidos.
Manifestó también a la asamblea que todos los que estaban allí debían
sentirse libres de expresarse, explicar, aclarar y argumentar sus
pensamientos; que las decisiones de la asamblea estarían basadas en la 
voluntad de la mayoría, sobre la base de la libertad expresada. 

Mariano Antonio Molas, representante del Partido Nacionalista, propuso
que el gobierno provisorio sea sustituido por una Junta conformada por
un presidente y cuatro vocales. También que el Paraguay mantenga buenas
relaciones con Buenos Aires y las demás provincias, y que se una
a ellas bajo una serie de condiciones, con el propósito de lograr una 
sociedad justa y libre para todos. Por último, Molas propuso que se 
suspendiera el juramento de fidelidad al Consejo de Regencia de España
hasta que se reuniera el Congreso General de las provincias que 
integraban el antiguo Virreinato del Río de la Plata, con Buenos Aires
a la cabeza, Puesto que ese Congreso General estaba pronto a celebrarse.
Durante el transcurso de las sesiones del Congreso, varias opiniones
fueron manifestadas, luego de lo cual 290 representantes votaron a favor
de las propuestas planteadas por Molas, y así quedó constituida la nueva
forma de gobierno, que fue la Junta Superior Gubernativa.

  \subsection*{Junta Superior Gubernativa}
  Los primeros gobiernos independientes surgieron prontamente.

El 16 de mayo de 1811 se constituyó un gobierno provisorio; el 17 de 
junio del mismo año, se reunió un Congreso que decidió como forma de gobierno
la Junta Superior Gubernativa, que estaba integrada por:

  \begin{itemize}
    \item \textbf{Presidente}: Fulgencio Yegros.
    \item \textbf{Vocales:} \hfill
      José Gaspar Rodríguez de Francia\\
      Pedro Juan Cavallero\\
      Presbítero Juan Francisco Xavier Bogarín\\
    \item \textbf{Vocal Secretario:} Fernando de la Mora
  \end{itemize}

La Junta de Buenos Aires pretendió mantener unido al antiguo Virreinato del
Río de la Plata y pidió que se enviara un representante del Paraguay para dar
respuesta al pedido. Las conclusiones del Congreso fueron comunicadas a la Junta 
Porteña mediante la célebre Nota del 20 de julio de 1811, en la que se reafirmó
la decisión categórica del Paraguay de ser libre e independiente de cualquier
poder externo. No cambiará de amo ni mudará unas cadenas por otras, expresaba 
el Paraguay negándose rotundamente a formar parte del extinto Virreinato del Río
de la Plata. Se firmó el Tratado del 12 de octubre de 1811, con los emisarios
argentinos, el general Manuel Belgrano y el doctor Vicente Anastasio de Echevarría.
El tratado tenía cláusulas económicas.

  \subsection*{Obras de la Junta Superior Gubernativa}
  La Junta Superior Gubernativa prohibió a los vecinos españoles ocupar cargos
públicos, con excepción de Juan Valeriano Zevallos; además, pidió a los vecinos 
de la capital del campo que juraran fidelidad al nuevo gobierno.

Obras de la Junta Superior Gubernativa.

La Junta Superior estaba con numerosas tareas que tenían que ver con el ordenamiento
del Paraguay como consecuencia de la independencia y observándose los peligros que 
representaban las fuerzas externas a su autonomía.

A pesar de esos problemas, se ocupó en promover la educación escolar y
promulgó el Bando del 6 de enero de 1812, que representa uno de los documentos 
más importantes de la historia del Paraguay. Este Bando es un decreto por el cual
el nuevo gobierno fijó sus planes, cuyos principales puntos fueron:

\begin{itemize}
    \item La enseñanza primaria obligatoria.
    \item Crear escuelas elementales y promover el desarrollo de la enseñanza superior
con la reapertura del Colegio Carolino.
    \item Crear las aulas de Matemática y Latinidad.
    \item Fundar la Sociedad Patriótica Literaria.
    \item Capacitar al personal docente.
    \item Mejorar la educación pública, pues las escuelas son el taller donde se
fomentan los grandes ciudadanos.
    \item Prohibir los castigos físicos en las escuelas y recomendar la persuasión
para ganar el apoyo de los estudiantes en la labor educativa.
    \item Organizar el ejército proyectando una academia militar.
    \item Intensificar la agricultura, la ganadería y el comercio.
    \item Promover una política social de alivio a los pobres y protección al nativo.
    \item Asegurar la libertad, la propiedad y los derechos del hombre.
\end{itemize}

  \subsection*{Educación}
  
Sin embargo, en esa época la educación estaba dirigida únicamente a los
varones al igual que los maestros. Mientras, las niñas se dedicaban a las
tareas del hogar. Por entonces, en las escuelas del Estado se implementó
el sistema lancasteriano, que consistía en que el mejor alumno de la clase
enseñaba a los alumnos que estaban más atrasados en el aprendizaje; y 
los alumnos más inteligentes debían ser instruidos en Geografia e Historia
de América, Historia Sagrada y Cronología.

Algunos de los maestros más renombrados de la época fueron el argentino
José Gabriel Téllez, quien desde 1802 se dedicaba a la enseñanza; Juan
Pedro Escalada fue otro gran maestro porteño, quien desde 1807 vivía en el
Paraguay y en muchas ocasiones daba clases particulares en su propia casa.

Otra de las órdenes de la Junta Superior Gubernativa fue la reapertura del
Real Colegio Seminario de San Carlos, que había sido clausurado por orden
del exgobernador Velasco. También fue fundada la Sociedad Patriótica
Literaria, la cual estaba encargada de velar la conducta de los estudiantes
y que estos no se dedicaran a perder el tiempo en juntas de juegos y demás
actividades inapropiadas.

Se dispuso también la creación de una academia militar, que sería la primera
en su tipo en el Río de la Plata. Se contrató también a un profesor de 
Buenos Aires para enseñar en una cátedra de Matemática, y en esa ciudad 
también se adquirieron numerosos materiales didácticos y libros para la
apertura de una biblioteca pública. Pero pese a las buenas intenciones en
materia cultural y educativa, muchos de estos proyectos no se realizaron, 
generalmente, por causa de las grandes dificultades económicas por las que
atravesaba el Paraguay en esos años.

  \subsection*{Economía}
  La Junta ordenó la libre circulación internacional por los ríos Paraguay
y Paraná, tanto para la navegación y para el comercio y transporte de cargas.
En cuanto a la agriculturas se promovió el mejoramiento de los cultivos,
especialmente del algodón que se utilizaba en la industria textil. Además,
fueron repartidas las tierras a las familias del interior del país.

  \subsection*{Misión de Nicolás Herrera}
  Es importante destacar que en los primeros meses de 1813 llegó al Paraguay 
el porteño Nicolás de Herrera con la misión especial de convencer al Paraguay
de que se una al gobierno federal y único de todas las Provincias de la
región, la cual estaba liderada por Buenos Aires.

Herrera expuso ante la Junta Superior Gubernativa las ventajas que obtendría 
la Provincia del Paraguay si se adhería a ese sistema federal. Y para ello
el gobierno del Paraguay tendría que enviar un diputado como representante 
ante el Congreso General de las Provincias Unidas del Río de la Plata que 
habían conformado el antiguo Virrreinato del Río de la Plata, que se 
llevaría a cabo en la cuidad de Buenos Aires.

El enviado porteño permaneció varios meses en Asunción esperando una respuesta 
positiva para su gobierno, puesto que este tema iba a ser estudiado en el 
próximo Congreso Nacional cuya fecha estaba próxima.

  \subsection*{Segundo Congreso Nacional, 1813}
  El 30 de septiembre de 1813 se convocó al Segundo Congreso Nacional, 
al que asistieron 1000 diputados de todos los puntos del país. Las sesiones 
estuvieron presididas por Pedro Juan Cavallero. El sistema que se manejó en
las sesiones fueron el del sufragio proporcional y universal, el más 
democrático utilizado hasta entonces. Puesto que antes de ese sistema,
los votantes en su mayoría eran comerciantes, estancieros, militares, etc.,
grandes propietarios, ricos y vecinos notables de la clase alta asuncena. Para
este Congreso se dictaminó el derecho al voto para todos los ciudadanos mediante 
elecciones libres.

Este congreso aprobó como nuevo gobierno al Consulado (1813-1814), con Fulgencio
Yegros y José Gaspar Rodríguez de Francia como cónsules. Aprobó asimismo
el Reglamento de Gobierno.

Entre sus disposiciones estaban:
  \begin{itemize}
    \item El gobierno reside en dos cónsules que se denominarán de la República del Paraguay.
    \item Cuidarán la conservación, seguridad y defensa de la República.
    \item La presidencia o el consulado será ejercida por cada cónsul alternativamente por un término de cuatro meses.
  \end{itemize}

El Paraguay fue el primer pueblo en el continente que declaró en forma enfática
su independencia absoluta de la metrópoli, desde entonces, se estampaba en los documentos
oficiales del Paraguay la frase: \``Primera República del Sur, el Paraguay, una e indivisible\''.
Además, el Congreso dispuso que el doctor Francia y Pedro Juan Cavallero redacten un documento 
para un reglamento de gobierno.

También el congreso declaró con respecto a las propuestas de Nicolás de Herrera, que no se 
enviaría ningún diputado al Congreso General de las Provincias Unidas del Río de la Plata. Era
claro que las intenciones de los porteños era recuperar al Paraguay y ejercer autoridad sobre
esta nación. Herrera nuevamente intentó convencer a los congresistas paraguayos acerca de 
las ventajas para el país esta nación de ingresar a ese sistema federal, e incluso
amenazó posteriormente con asfixiar económicamente al Paraguay, en caso de que no se enviara un
diputado al Congreso General. Sin embargo, el Congreso Nacional se mantuvo firme en su negativa.

  \subsection*{El Reglamento de Gobierno, 1813}
Con el fin de cumplir con la disposición del Congreso, doctor José Gaspar Rodríguez
de Francia y Pedro Juan Cavallero elaboraron y presentaron un documento el 12 de
octubre de 1813, conocido como Reglamento de Gobierno Este reglamento constituyó 
la ley fundamental de la República independiente y estipulaba los siguientes puntos:
la finalización del mandato de la Junta y la proclamación de una nueva forma
gubernativa: la República, la primera instituida en América del Sur. A partir
de esa fecha, con esta denominación, el Paraguay dejó de ser una provincia. Siguiendo
el modelo de la antigua república romana, el gobierno estaría a cargo de dos 
cónsules: Gaspar Rodríguez de Francia y Fulgencio Yegros. Ambos gobernarían en turnos
de cuatro meses cada uno. Y su papel fundamental debía ser la conservación, defensa
y seguridad de la República del Paraguay.

  \subsection*{El primer consulado}

A pesar de que los dos cónsules tenían las mismas prerrogativas y poderes, el doctor 
Francia fue el que verdaderamente dirigió el gobierno en este periodo. Francia
fue el primero en dirigir el Consulado, luego Yegros y, por último, Francia nuevamente.

Para los habitantes españoles del Paraguay, el régimen del Consulado representó graves
problemas, puesto que los cónsules tenían el objetivo de debilitar a la minoría española
y extinguir su influencia política y económica mediante la adopción de varias medidas
represivas e impositivas.

Antes de abandonar el Paraguay, Herrera intentó por última vez conseguir del Consulado su
aceptación  a la idea de federación de las Provincias y el envío de un diputado. El doctor
Francia se negó a aceptar el acuerdo y en represalia el gobierno de Buenos Aires impuso 
nuevos y más pesados impuestos sobre los cargamentos paraguayos que llegaban a los puertos argentinos.

Finalizado el Congreso General de las Provincias Unidas, Herrera trató nuevamente de
realizar algún acuerdo con el Paraguay con la finalidad de que este gobierno le facilitara
tropas o dinero. Herrera propuso que firmaran un tratado de comercio y alianza entre 
Paraguay y Buenos Aires en donde manifestaran públicamente las buenas relaciones de
fraternidad y ayuda mutua.

Sin embargo, el cónsul Francia respondió que el Paraguay no necesitaba firmar ningún tipo
de tratado para mantener las buenas relaciones con sus vecinos y mantener la libertad 
y fraternidad. Por lo tanto, Francia le reiteró la amistad del Paraguay hacia Buenos Aires
y la independencia total del Paraguay. Finalmente, Herrera tuvo que ir a Buenos Aires
sin haber obtenido alguna respuesta favorable por parte del Paraguay.

      \pagebreak
      \hrule
      \vspace{9pt}
      \section*{Ejercicios}
      \subsubsection*{¿Quiénes formaron parte del triunvirato y cuáles eran las leyes?}
      \subsubsection*{¿Quién era la persona más influyente en esa época (en el Paraguay)?}
      \subsubsection*{Cita cuáles fueron los principios del gobierno de Francia.}
      \itemize{
      \item *\\
      \item *\\
      \item *\\
      \item *\\
      }
      \subsubsection*{Realiza un resumen del reglamento de gobierno de 1813.}

      \pagebreak
      \hrule
      \vspace{9pt}
      \section*{Bibliografía}
      \textit{Libro de Historia y Geografía 2do Plan Común}

      \textit{https://economipedia.com/definiciones/gobierno-de-gaspar-francia.html}

      \textit{https://www.worldhistory.org/trans/es/2-2204/paraguay/biografia-de-jose-gaspar-rodriguez-de-francia/}
  \end{document}
