\documentclass{article}
\usepackage[a4paper, total={6in, 8in}]{geometry}
\usepackage{graphicx}
\usepackage{mathptmx}

\begin{document}
\begin{titlepage}
  \begin{center}
      \vspace*{1cm}

      \Huge
      \textbf{La Revolución Industrial}

      \vspace{0.5cm}
      \LARGE
      \vspace{1.5cm}

      \textbf{Jonathan Bray, Ever Ortega, }

      \vfill

      Historia y Geografía

      \vspace{0.8cm}


      \Large
      Colegio Técnico Nacional\\
      Especialidad: Informática\\
      Profesora: Mirian Montanía
      03/04/2024

  \end{center}
\end{titlepage}

  \Large
  \noindent Primera Revolución Industrial \dotfill2\\
  Causas fundamentales de la Primera Revolución Industrial \dotfill2\\
  Segunda Revolución Industrial \dotfill2\\
  Tercera Revolución Industrial \dotfill3\\
  Liberalismo económico \dotfill3\\
  Consecuencias sociales de la Revolución Industrial \dotfill4\\
  Nota Histórica\dotfill5

  \normalsize
  \pagebreak
  \section*{\large Causas, características y consecuencias de la Revolución Industrial en las diferentes etapas de su proceso}
  \hrule
  \vspace{0.5cm}
  \section*{Siglo XVIII. Primera Revolución Industrial}
  
  La Revolución Industrial es un proceso de cambio social, económico y cultural, desarrollado en Gran Bretaña y que se difundió más tarde en Europa y en otras partes del mundo.

  La primera Revolución Industrial se produjo en el siglo XVIII, en Inglaterra y la segunda, en el siglo XIX. Europa pasó de un mundo rural a un mundo industrial, de un taller artesano a la fábrica, del trabajo manual a la mecanización.

  \subsection*{Causas fundamentales de la Primera Revolución Industrial}

  \begin{itemize}
      \item El aumento de la población debido al descenso de la mortalidad.
      \item La mejora en la producción agrícola gracias a las nuevas técnicas de cultivo.
      \item Los nuevos inventos aplicados a la producción de bienes.
  \end{itemize}

  El invento que más influyó fue la máquina de vapor (1769), creado por James Watt (1736-1819), que se aplicó a la industria, a la minería y al transporte. Fue la fuente de energía básica de la Revolución Industrial.

  La industria textil se transformó gracias al telar mecánico de Edmund Cart-\\wright (1743-1823). El algodón, más barato y abundante, desplazó a la lana.

  La industria siderúrgica tuvo un gran desarrollo. El carbón mineral desplazó al carbón vegetal y se utilizó en las nuevas fundiciones: los altos hornos multiplicaron la producción de hierro, necesario para renovar los utensilios agrícolas y para las nuevas máquinas.

  La nueva forma de fabricación llevaba consigo nuevas formas de trabajo; el pequeño taller artesanal dio paso a la fábrica, donde se concentraban trabajadores y máquinas.

  La revolución de los transportes se inició en 1830. George Stephenson (1781-1848) inventó la locomotora aplicando la máquina de vapor a un vagón sobre raíles.

  En 1830, el primer ferrocarril hizo su recorrido entre Liverpool y Manchester. A comienzos del siglo XIX también se aplicó la máquina de vapor al barco. Los barcos de vapor se utilizaron en los grandes ríos como el Misisipi y en 1848 atravesaron el Atlántico.

  \section*{Segunda Revolución Industrial}
  La industrialización fue extendiéndose desde Inglaterra hacia el continente. Francia y Bélgica fueron los primeros países europeos en industrializarse. A partir de 1870, se desarrolló el Gran Capitalismo o Segunda Revolución Industrial. Esta etapa se caracterizó por el desarrollo de nuevas industrias y fuentes de energía, por la creación de grandes empresas y por la extensión de las industrias a nuevos países.

  \section*{\large Nuevas fuentes de energía}
  \subsection*{\normalsize Electricidad}
  A finales del siglo XIX Thomas Edison inventó la dinamo generadora de corriente directa para producir la electricidad y la lámpara incandescente o bombilla.
  Otros inventos, como el transformador o alternador, permitieron utilizar la electricidad para el alumbrado como fuerza motriz de las máquinas, para los transportes.

  \subsection*{\normalsize Petróleo y sus derivados}
  Se utilizaron como combustibles para los automóviles fabricados a partir de los inventos de Daimler y Benz (el motor de explosión) y de Rudolf Diesel (el motor de combustión interna). También se utilizó para otros transportes, para barcos y aviones. Los hermanos Orville y Wilbur Wright realizaron el primer vuelo de 1903.

  Entre las nuevas industrias se destacaron la química (colorantes, productos farmacéuticos, abonos) la eléctrica y sus aplicaciones a las telecomunicaciones (telégrafo, teléfono, radio), la automovilística y la metalúrgica (acero inoxidable, cobre, aluminio etc.).

  Las nuevas empresas necesitaban invertir más cantidad de capital para realizar sus actividades: los bancos financiaron las empresas. En esta revolución, se destacaron otros países, como Alemania, Japón y Estados Unidos, este último se convirtió en la primera potencia industrial.


  \section{Tercera Revolución Industrial}
  La tercera Revolución Industrial ha comenzado tras la Segunda Guerra Mundial. Aparecen nuevas fuentes de energía, como la nuclear o las renovables, en especial la eólica y la solar, estas han ido ganando protagonismo en el contexto de una economía cada vez más preocupada por el crecimiento sostenible. (Fernández, 2012, pp. 66-67)

  Nuevos sectores, como la informática, la robótica, las telecomunicaciones o la industria aeroespacial han tomado el relevo a los altos hornos o a la construcción naval. Y la nueva economía-mundo, cada vez más globalizada, se fue abriendo paso, derribando las antiguas fronteras nacionales y conformó un nuevo panorama en la hegemonía de Occidente, que se vio amenazada por países emergentes de una dimensión demográfica y económica que fue convirtiendo el Estado-nación tradicional en un recuerdo de otro tiempo.

  \subsection*{Liberalismo económico}

  Con la Revolución Industrial se impuso una nueva teoría: el liberalismo econó-\\mico. Adam Smith (1723-1790) fue un pensador y economista escocés que elaboró esta teoría que constituyó la base doctrinal del capitalismo y lo expuso en su obra \textit{La riqueza de las naciones} (1776). Algunas ideas principales de la obra son:

  \begin{itemize}
      \item El trabajo es el origen de la riqueza y se realiza en función de un interés particular.
      \item La ley de la oferta y la demanda regula los precios de los productos y los salarios de los trabajadores.
      \item Cada persona persigue su lucro y beneficio personal, debe tener libertad para ejercer el trabajo y las actividades económicas que le permitan obtenerlo\item Este interés personal beneficiará a la colectividad.
      \item El Estado no debe intervenir en las cuestiones económicas, debe dejar que se regulen libremente mediante la oferta y la demanda, y la iniciativa de los ciudadanos.
  
  \end{itemize}

  \subsection*{Consecuencias sociales de la Revolución Industrial}
  El liberalismo económico dio lugar a condiciones de trabajo muy duras, pues el Estado no se inmiscuía en cuestiones económicas, no había legislaciones que regularan el contrato entre el trabajador y el capitalista: Así, el afán de producir más y en menos tiempo condujo a largas jornadas de trabajo, de 12 a 14 horas, los salarios eran bajos, se contrató a mujeres y niños por salarios menores. Sus viviendas eran miserables y caras, generalmente, eran sótanos o desvanes que no conocían la luz del sol ni poseían defensa contra el calor asfixiante del verano y el frío gélido del invierno, además de la privación absoluta de cualquier sistema de protección social; cualquier enfermedad o accidente suponía una pérdida total del ingreso y la miseria más completa para el obrero y su familia. Los obreros formaron el proletariado industrial integrado por la clase baja.

  Los nuevos empresarios o dueños de las fábricas formaron una burguesía industrial que poseía grandes riquezas. Fueron los nuevos capitalistas que se integraron en la clase alta.

  El crecimiento de las industrias en las ciudades y la oferta de los puestos de trabajo atrajeron a una población creciente. Las viviendas de los obreros y las condiciones de trabajo en las ciudades fueron malas. Esto provocó quejas y motines. Los primeros actos de protestas consistieron en la destrucción de las máquinas, acción que se denominó ludismo, ya que fue un obrero irlandés, Ned Ludd, el primero en realizarlas en los años 1811 y 1812.
  
  A partir de 1925 se crearon en Inglaterra las primeras \textit{trade unions} (sindicatos), asociaciones de trabajadores por oficios que pretendían mejorar las condiciones de trabajo de los obreros. Además, aparecieron doctrinas sociales que abogaban por el mejoramiento de las condiciones de vida y de trabajo de los obreros, como el anarquismo, el socialismo utópico y el socialismo científico.

  Con la formación de los sindicatos, las asociaciones internacionales y las doctrinas sociales, fueron mejorando las condiciones de trabajo de los obreros: adquisición de seguros sociales, reducción de las horas laborales, vacaciones pagas y establecimiento de leyes protectoras.

  \subsection*{Otras consecuencias}
  La vida cotidiana sufrió cambios, ya que la población se concentró en las ciudades, como resultado de la migración del campo. Los burgueses gozaron de muchas comodidades (mejores ropas, alimentos variados que provenían de lugares lejanos, luz de gas y después eléctrica) y diversiones, realizaban viajes turísticos, disfrutaban de la cultura porque tenían más preparación y dinero para asistir al teatro, conciertos y adquirir pinturas de famosos. La cultura se vio enriquecida con el fonógrafo y el cinematógrafo, las noticias llegaron más rápidamente; numerosos periódicos circulaban, estos eran leídos por los burgueses, quienes estaban más preparados para hacerlo.

  Los medios de transporte se diversificaron (trenes, barcos y automóviles, más tarde los aviones), el mundo estuvo más interconectado, se desarrollaron numerosas industrias y servicios, también creció el comercio.

  \fbox{\begin{minipage}{\textwidth}
      NOTA HISTÓRICA\\
      En esta fábrica trabajan mil quinientas personas, y más de la mitad tienen menos de quince años. La mayoría de los niños están descalzos. El trabajo comienza a las cinco y media de la mañana y termina a las siete de las tarde, con altos de media hora para el desayuno y una hora para la comida. Los mecánicos tienen media hora para la merienda, pero no los niños ni los otros obreros (...).
      Cuando estuve en Oxford Road, Manchester, observé la salida de los trabajadores cuando abandonaban la fábrica a las doce de la mañana. Los niños, en su casi totalidad, tenían aspecto enfermizo; eran pequeños, enclenques e iban descalzos. Muchos parecían no tener más de siete años. Los hombres en su mayoría de dieciséis a veinticuatro años, estaban casi tan pálidos y delgados como los niños. Las mujeres eran las de apariencia más saludable, aunque no vi ninguna de aspecto lozano (...). Aquí vi, o creí ver, una raza degenerada, seres humanos achaparrados, debilitados y depravados, hombres y mujeres que no llegarán a ancianos, niños que nunca serán adultos sanos. Era un espectáculo lúgubre (...) (Turner Thackrah, 1832).
      \end{minipage}}
      \begin{center}
          \vspace{4cm}
          \includegraphics[scale=0.7]{image1.jpg}
      \end{center}
      \pagebreak
      \section*{Bibliografía}
      \textit{Libro de Historia y Geografía 2do Plan Común}

      \textit{https://economipedia.com/definiciones/primera-revolucion-industrial.html}

      \textit{https://www.worldhistory.org/trans/es/2-2204/los-10-mejores-inventos-de-la-revolucion-industria/}
  \end{document}

