\documentclass{article}
\usepackage{amsmath, amsfonts, amssymb}
% \usepackage{mathptmx}

\title{Cosas que están en el libro de ciencias y que pueden salir en el examen}
\author{Ever Ortega}
\date{\today}

\begin{document}

\section{Sistema nervioso}
  \textbf{Hipófisis:} 
  Glándula situada en la base del cerebro y que recibe órdenes del hipotálamo
  y actúa sobre otras glándulas para que produzcan sus propias hormonas.

  \textbf{Las funciones de relación se llevan a cabo con la participación de 
  los sistemas nervioso, óseo, muscular, articular, sensorial y endocrino}

  El sistema nervioso se compone del \textbf{encéfalo, la médula espinal y los
  nervios} que se extienden por todo el cuerpo

  La \textbf{meninges} protege a todo el sistema nervioso central, tiene \textbf{3 capas}, y en el medio está el \textbf{fluido cerebroespinal}.

  Materia que cubre la corteza cerebral: \textbf{Materia gris y Materia blanca}.
\section{Sistema musculoesquelético}
Rama de la ciencia que estudia la anatomía de los huesos: \textbf{osteología}

\subsection{Sistema óseo}
\textbf{Clasificación de los huesos:} largos(fémur, cúbito, radio, etc,),
que sirven de palanca; cortos (huesos de la muñeca); planos (occipital,
temporal, parietal, etc.), 
que sirven de protección.

\vspace{0.5cm}
Largos: Están formados por dos extremos en los que se encuentran las dos
superficies articulares llamadas \textbf{epífisis superior} y
\textbf{epífisis inferior}

\vspace{0.5cm}
Cortos: Formado por \textbf{tejido esponjoso y tejido laminar compacto} en su
exterior (huesos del carpo, muñeca)

\vspace{0.5cm}
Planos: Su sinónimo es \textbf{diploe}. Está formada por dos capas de
tejido compacto y una de tejido esponjoso en orden trivial.

\vspace{0.5cm}
Irregulares: Son tan kachiãi que no son ninguno de los anteriores, es 
otro nombre para memorizar.

\vspace{0.5cm}
\textbf{Estructura ósea}
\vspace{0.5cm}

\textbf{Capas del hueso: } Periostio, sustancia ósea y médula ósea.

\textbf{Periostio:} Envuelve al hueso en toda su extensión salvo en los extremos,
donde está cubierta de cártílago.

\textbf{Sustancia ósea:} Tejido óseo, vasos sanguíneos y nervios.
Hay porciones en que las células están más densificadas 
(más juntas), eso es el tejido compacto.

\textbf{Médula ósea:} Puede ser roja o amarilla. La roja tiene función
\textbf{hematopoyética} (producir sangre); y la amarilla es de
consistencia adiposa (grasa).


\textbf{Osteocitos:} Células del hueso.

\textbf{Osteoblastos:} Conjunto de osteocitos.

\subsection{El esqueleto humano}
\subsubsection{Lista de afirmaciones random que pueden salir en el examen:}
\begin{itemize}
  \item Hay 206 huesos
  \item Hay 26 vértebras
  \item Hay 12 pares de costillas (arcos óseos)
  \item La cara está integrada de 14 huesos.
  \item La columna vertebral se divide en 5 secciones.
  \item Los primeros 7 pares se unen al esternón.
  \item Los siguientes 2 por un cartílago al esternón.
  \item Los últimos 2 "flotan".
  \item Entonces existe un par de costillas que el libro simplemente ignora (7
    + 2 + 2 = 11 != 12)

\end{itemize}

\subsubsection{Descripción que capaz salga en el examen}
La cabeza se mueve por medio de la \textbf{primera vértebra cervical y el
occipital}. Unido a la primera vértebra cervical se encuentra el
\textbf{atlas o segunda vértebra cervical}.

La cabeza se divide en cráneo y cara, el cráneo contiene al cerebro y la
cara es lo que sobra de la cabeza.

El cráneo se divide en \textbf{bóveda y base}.

\begin{itemize}
  \item Bóveda: porción del fronta y occipital, los parietales y los temporales.
  \item Base: Separa a los huesos de la cara y el cráneo, formado por \textbf{esfenoides,
    etmoides, porciones del frontal y el occipital.}

\end{itemize}

\subsubsection{Hioides}
El hueso hioides es un tkk'i que flota en el cuello, es la manzanita.

\subsection{Columna vertebral}
\subsubsection{Lista de nombres random:}

\begin{itemize}
  \item Apófisis: la vértebra tiene tres puntas, la punta que sobresale hacia
    afuera es llamada \textbf{apófisis dorsal} por alguna razón, las otras 
    dos son llamadas \textbf{apófisis transversales}.

\end{itemize}

\subsubsection{Extremidad superior}
Consta de \textbf{hombro, brazo, antebrazo y mano}. 

\begin{itemize}
  \item \textbf{Hombro = Cintura escapular}: formado por \textbf{omóplato = escápula y la clavícula}. 
  \item Brazo: formado por el \textbf{húmero}, se une con la \textbf{cintura escapular} y el \textbf{codo}.
\item Antebrazo: formado por \textbf{cúbito} y \textbf{radio}. Un dato curioso es que el cúbito es 
  más grueso que el radio hacia el codo, pero hacia la muñeca es alrevés (bastante xd).
\end{itemize}

\subsubsection{Mano}
Tres segmentos: \textbf{carpo, metacarpo y falanges}.
\begin{itemize}
  \item Carpo: \textbf{escafoides, semilunar, piramidal y pisiforme; trapecio, trapezoide, grande y ganchoso}.
  \item Metacarpo: Formado por los \textbf{metacarpianos}.
  \item Falanges: \textbf{falange, falangina y falangueta}.
\end{itemize}

\subsection{Extremidad inferior}
Unido al tronco por el \textbf{cinturón pélvico = ilíacos + sacro}.

Se divide en:
\begin{itemize}
  \item Cadera: \textbf{huesos coxales (ilíacos) + cabeza de fémur}
  \item Muslos: \textbf{Fémur}. Se articula arriba con la \textbf{cadera} y abajo con la \textbf{tibia y peroné}
\item Piernas: \textbf{Tibia + Peroné}. Se artivula arriba con \textbf{fémur} y abajo con \textbf{tarso}.
\item Pies: \textbf{Tarso, metatarso y falanges (dedos)}
\item Tarso = \textbf{astrágalo + calcáneo + escafoides + cuboides + 3*huesos-cuneiformes}
\item Metatarso = \textbf{5 * metatarsianos}, que se articulan con el tarso y existe 
  una biyección de ellos con las falanges.
\end{itemize}
\subsection{Sistema muscular}

Músculos que suelen salir en selección múltiple:

\textbf{Cuádriceps:} Músculo \textbf{anterior} a la pierna.

  \textbf{Isquiotibiales: } Atrás del cuarto (muslo) o Músculo \textbf{posterior} a la pierna.

  \textbf{Bíceps: } Músculo \textbf{anterior} al brazo.

  \textbf{Tríceps: } Músculo \textbf{posterior} al brazo.

  Las fibras musculares se unen no como cabos, porque no están enroscadas, pero sí se parecen a 
  los cabos en es sentidon que las fibras no se unifican a lo largo del músculo.

  \textbf{Atrofia muscular:} Cuando los músculos dejan de moverse, se atrofian.

  \textbf{Hipertrofia muscular:} Cuando los músculos se ven obligados a aer más fuertes, crecen.
  
  \textbf{Hipotonía e hipertonía muscular:} Cuando estamos quietos, los músculos siguen 
  tensos para formar el tono muscular, hay hipotonía muscular si falta tono muscular
  e hipertonía muscular en caso constrario.

\section{Sistema articular}
La \textbf{artrología} es la ciecia que estudia a las articulaciones.

Hay \textbf{3} tipos de articulaciones: \textbf{sinartrosis, anfiartrosis y diartrosis}

\begin{itemize}
  \item Sinatrosis: Articulaciones inmóviles. Un ejemplo son los huesos del cráneo.
  \item Anfiartrosis: Se une con \textbf{fibrocartílago} a los otros huesos. Se pueden mover pero no hay músculos que los mueven directamente. Un ejemplo son las costillas.
  \item Diartrosis: Consta de: \textbf{cuerpo articular, cápsula articular, cavidad articular y líquido articular}. 

    \textbf{Lo demás de articulaciones estudiar estrictamente de la página 51 del libro}.
\end{itemize}

\section{Sistema Sensorial}
\subsection{Datos random que pueden salir:}
\begin{itemize}
\item Los órganos de los sentidos se clasifican en \textbf{sentidos
  físicos y sentidos químicos. La vista, el oído y el tacto;
y el olfato y el gusto respectivamente.}
\end{itemize}
\subsection{Órgano del tacto}
\subsubsection{Capas de la piel}
\itemize{
  \item Epidermis
  \item Dermis
  \item Hipodermis ó \textbf{Tejido celular subcutáneo}
}

\subsubsection{\textbf{Epidermis}}
En la epidermis se encuentran los \textbf{melanocitos},
que son células que producen la \textbf{melanina},
que es el compuesto que caracteriza al color de la piel,
mientras más melanina haya, más oscura se verá la piel y viceversa.

\subsubsection{Lista de órganos random cuyas definiciones se conocen
pero que no se recuerden tan fácilmente:}
\begin{itemize}
  \item Epidermis
  \item Dermis
  \item Hipodermis o \textbf{Tejido subcutáneo}
  \item Poro
  \item Tejido adiposo
  \item Glándula sudorípara
\end{itemize}

\subsubsection{Funciones de la piel como órgano:}
\begin{itemize}
  \item Protección
  \item Sensibilidad: Las cuatro sensaciones fundamentales: \textbf{dolor, tacto,
    presión (no la atmosférica) y temperatura}
  \item Termorregulación: Transpiración y \textbf{piloerección}
    (los pelos de punta para aislarse de un clima frío)
  \item Metabolismo del agua: la piel colabora con otros 
    órganos, por ejemplo el riñón: si hay mucha
    transpiración, entonces el riñón produce orina más 
    concentrada para evitar morir de sed.
\end{itemize}
\end{document}
